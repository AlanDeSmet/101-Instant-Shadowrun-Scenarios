\begin{scenario}{Take This Job...}
	{Extraction}
	{A voluntary extraction, complicated only by a cranial bomb on the target.}
	{Alan De Smet}
	{2017-01-22}
	{}

\johnson{Johnson}{Any megacorporation} There is nothing special about the Johnson.  The job is straightforward: extract a Grace Yao, currently in the employ of Fuchi Industrial Electronics.  Ms Yao is very eager to lave Fuchi's employ. Unfortunately Fuchi has a strong incentive program, in the form of a cranial bomb in her head.

\synopsis Yao is one of the best bioengineers in the world, specializing in in bioware, thus Johnson's eagerness to bring her into their employ.  Fuchi has gone to great lengths to keep her in their employ. She has an area cranial bomb and tracker embedded in her.  Unfortunately for Fuchi, this has made her eager to find new employment.

Fuchi is not currently aware of Yao's plans, so she has some freedom outside of Fuchi facilities. However, to protect against the risk of hostile acquisition, she receives the sort of protection of the UCAS president receives: all movements are planned and researched in advance, a small scout team goes in early, and Yao is surrounded by several heavily cybered guards.  Yao typically travels by armoed car with an additional escort vehicle, with similar guards.

The cranial bomb has both an active and passive mode.  If an expected signal is not received every 5 minutes, it enters warning mode.  In warning mode, Yao suffers from a strong headache; she knows what this means.   If the signal is later received, the bomb deactivates, although it takes Yao a half hour to fully recover.  If the signal is not received within another 30 minutes, the bomb detonates.  In addition, there is a signal that can be sent to immediately trigger the warning mode or detonation.

While Fuchi is keen to discourage Yao from quitting, and to deny her to any competitors, they want to avoid killing her unnecessarily.  Policy is for the first sequence to be sent 5 minutes after contact with her is lost.  Inside of of Fuchi facilities, contact is nearly constant courtesy of tracking sensors. In public there are occasional shielded areas and unexpected jamming, so contact is manually maintained by her handlers.  At least one handler will have a devices which can broadcast the first signal over a short range, but policy is to try and contact HQ and have them broadcast it, as the Fuchi's broadcast is generally stronger.

Detonation is entirely controlled at HQ, with several people needing to sign off. Policy is that 30 minutes after the first signal is sent, assuming contact has not been re-established, the detonation signal will be sent.  However, given clear evidence of Yao intending to escape, the detonation signal will be triggered as quickly as possible, typically within 10 minutes.

Yao's home is in a luxury apartment complex, penthouse floor.  The penthouse and floor below are owned by Fuchi. The floor has several luxury apartments used for lesser, but still valuable, Fuchi employees and guests, but is half dedicated to housing and serving as an HQ for security for Yao and the other Fuchi tenants.

Yao's office is superficially a Fuchi funded startup space, including security and shared space.  However, underground it houses some of Fuchi's bioware research labs.  External security appears to be average quality, but is better equipped than one might expect. Below the ground, security is quite high.

\end{scenario}
